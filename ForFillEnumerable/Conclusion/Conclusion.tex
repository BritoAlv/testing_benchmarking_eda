% add --shell-escape to pdflatex arguments.
% add following key to have keyboard shortcuts
%{
%    "key": "shift+b",
%    "command": "commandId",
%    "when": "editorTextFocus"
%},
%{
%"key": "shift+B",
%"command": "editor.action.insertSnippet",
%"when": "editorLangId == latex && editorTextFocus",
%"args": {
%    "snippet": "\\textbf{${TM_SELECTED_TEXT}$0}"
%}
%}

\documentclass[14pt]{extarticle}
\usepackage[left=1cm , right = 2cm, top=2cm]{geometry}
\usepackage{helvet}
\usepackage{parskip}
\usepackage{amsmath}
\usepackage{amssymb}
\usepackage{graphicx}
\usepackage[spanish]{babel}
\usepackage[dvipsnames]{xcolor}
\usepackage{tcolorbox} % above of the svg package
\usepackage{svg} 
\usepackage{hyperref}
\usepackage{minted}
\usepackage{csvsimple}
\renewcommand{\sfdefault}{lmss}  % este activa la letra lmss
\renewcommand{\familydefault}{\sfdefault} % este activa la letra lmss
\sffamily % este activa la letra lmss
%\hyperlink{page.2}{Go to page 2}
%\newpage
%text on page 2
%\begin{figure}[htbp]
%  \centering
%  \includesvg{plot.svg}
%  \caption{svg image}
%\end{figure}

%\begin{minted}{csharp}
%    // single comment
%    \end{minted}

%\begin{align}
%    \frac{d}{dx} \ln x &= \lim_{h\to 0} \frac{\ln(x+h) - \ln x}{h} \\
%    &= \ln e^{1/x} &&\text{How this follows is left as an exercise.}\\
%    &= \frac{1}{x} &&\text{Using the definition of ln as inverse function}
%   \end{align}


\begin{document}



@BritoAlv



\begin{tcolorbox}[colback=blue!5!white,colframe=blue!75!black, title = Fastest way to fill an array with 1's]

    Dado un array, digamos que necesitamos rellenarlo con un valor prefijado como $1$. Se obtuvieron los siguientes resultados:
\end{tcolorbox}

\begin{figure}[htbp]
    \centering
    \includesvg{plot.svg}
    \caption{Plot de los resultados del bench-mark para distintas entradas}
\end{figure}

Primero se usaron los tres siguientes algoritmos para hacer un benchmark:

\begin{minted}{csharp}
    // la peor idea 
    data = Enumerable.Repeat(1, N).ToArray();
\end{minted}

\begin{minted}{csharp}
    // lo que se le ocurriría a una persona en su sano juicio
    for (int i = 0; i < data.Length; i++)
    {
        data[i] = 1;
    }
\end{minted}

\begin{minted}{csharp}
    // librería de c#
    Array.Fill(data, 1);
\end{minted}

Como se puede observar en el plot, la solución usando \texttt{IEnumerable} no es la más optimizada, se pensaría que el  \texttt{For} es la mejor opción, pués no, resulta que el método \texttt{Fill} se las arregla para ser más rápido. ¿ A qué se debe esto?.

\section{IL-For}
Primero el \texttt{For} es linear, y bastante estable como se puede observar en los plots.

\begin{figure}[htbp]
    \centering
    \includesvg{plo.svg}
    \caption{Plot de los resultados del bench-mark para distintas entradas}
\end{figure}

El siguiente for es traducido a \texttt{IL} por C\# obteniéndose:
\begin{minted}{csharp}
int[] data = new int[5];
for (int l = 0; l < data.Length; l++)
{
    data[l] = 3;
} 
\end{minted}

\begin{minted}{csharp}
    // IL CODE.
    // push 5 en el stack como int32.
    IL_0000: ldc.i4.5

    // pop del stack el tamaño del array, crear el array, y 
    // push en el stack una referencia al array.
    IL_0001: newarr 

    // pop lo que hay en el stack y guardarlo en la variable local 0.
    IL_0006: stloc.0

    // push 5 en el stack como int32.
    IL_0007: ldc.i4.0 // poner 0 en el stack como int32.

    // pop lo que hay en el stack y guardarlo en la variable local 1.
    IL_0008: stloc.1 
    // sequence point: hidden
    IL_0009: br.s IL_0013 // instrucción de branch.
    // loop start (head: IL_0013)
    
    // push variable local 0 en el stack.
    IL_000b: ldloc.0 

    // push variable local 1 en el stack.
    IL_000c: ldloc.1

    // push 3 en el stack como int32.
    IL_000d: ldc.i4.3

    // pop el 3, pop el l, pop el array y asignar el valor.
    IL_000e: stelem.i4

    // push variable local 1 en el stack.
    IL_000f: ldloc.1

    // push 3 en el stack como int32.
    IL_0010: ldc.i4.1
    
    // pop 3, pop local 1 y push el resultado en el stack.
    IL_0011: add
    
    // pop lo que hay en el stack y lo pone en variable local 1.
    IL_0012: stloc.1

    // push varible local 1 (l) en el stack
    IL_0013: ldloc.1
    
    // push el array en el stack
    IL_0014: ldloc.0
    
    // pop el array,calcula la longitud y la push en el stack
    IL_0015: ldlen
    
    // pop lo que hay en el stack, lo convierte a int32 y lo push nuevamente.
    IL_0016: conv.i4
    
    // push los dos primeros valores del stack y jump to IL_OOOb
    // si es verdadero, si no acaba el loop.
    IL_0017: blt.s IL_000b
// end loop
\end{minted}


Entender lo que hace este \texttt{For} es importante púes como se observa, inicializa la variable $l$ y después repetidamente asigna el valor de $l$ y comprueba si puede continuar. Esto es $O(n)$, en fin si hay alguna implementación mejor deberá realizar menos instrucciones, \textbf{eso es lo que hace Fill}. 

\section{SIMD}
La explicación está en el uso de \textbf{SIMD}, que significa Single Instruction Multiple Data, esto consiste en nuevas instrucciones que poseen los procesadores, cuya idea es aplicar la misma instrucción a various bloques de memoria de un tiro, por ejemplo en 256 bits se pueden guardar 8 floats ocupando 32 bits cada uno, si con una instrucción accedemos a ese bloque de 256 bits, y con otra indicamos que queremos asignarle el valor de $1$ a cada valor, realizamos las instrucciones de 8 en 8 vs a 1 a 1, he aquí la optimización. En esto se basan los otros métodos \texttt{Copy} también usados, que prácticamente son iguales de rápido que *Fill*. 

Verificando la rapidez de las instrucciones de \textbf{SIMD} ,usando el lenguaje C++,  a través de una implementación del DotProduct que es un ejemplo típico se obtuvo que:

\begin{figure}[htbp]
    \includesvg{simdvsregularforcpp.svg}
    \caption{La entrada es potencias de 2, ahora en C++}
\end{figure}




\end{document}

